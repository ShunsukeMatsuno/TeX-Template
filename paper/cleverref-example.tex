\documentclass[11pt]{article}
\usepackage{paper}

\title{A Demonstration of Cleveref Cross-Referencing}
\author{Shunsuke Matsuno}
\date{\today}

\begin{document}

\maketitle

\section{Introduction}

This document demonstrates the use of \texttt{cleveref} for cross-referencing various environments. The \texttt{cleveref} package automatically determines the appropriate reference format and excels at handling multiple references elegantly.

\section{Mathematical Environments}

\subsection{Single References}

Let's start with basic mathematical statements:

\begin{theorem}\label{thm:fundamental}
Every bounded sequence in $\R$ has a convergent subsequence.
\end{theorem}

\begin{lemma}\label{lem:helper}
If $\{a_n\}$ is a Cauchy sequence, then it is bounded.
\end{lemma}

\begin{definition}\label{def:cauchy}
A sequence $\{a_n\}$ is called \emph{Cauchy} if for every $\varepsilon > 0$, there exists $N \in \N$ such that for all $m, n \geq N$, we have $|a_m - a_n| < \varepsilon$.
\end{definition}

We can reference these individually: \Cref{thm:fundamental} is the Bolzano-Weierstrass theorem, \Cref{lem:helper} is a standard result, and \Cref{def:cauchy} defines Cauchy sequences.

\subsection{Multiple References of Same Type}

Here are additional results:

\begin{theorem}\label{thm:completeness}
Every Cauchy sequence in $\R$ converges.
\end{theorem}

\begin{theorem}\label{thm:existence}
Every continuous function on a compact set attains its maximum and minimum.
\end{theorem}

\begin{lemma}\label{lem:bounded}
Every convergent sequence is bounded.
\end{lemma}

\begin{lemma}\label{lem:continuity}
A function is continuous if and only if it preserves limits of sequences.
\end{lemma}

Now we can demonstrate multiple references:
\begin{itemize}
\item Two theorems: \Cref{thm:fundamental,thm:completeness}
\item Three theorems: \Cref{thm:fundamental,thm:completeness,thm:existence}
\item Two lemmas: \Cref{lem:helper,lem:bounded}
\item Multiple lemmas: \Cref{lem:helper,lem:bounded,lem:continuity}
\end{itemize}

\subsection{Mixed Mathematical Environment References}

We can reference different mathematical environments together:
\begin{itemize}
\item Theorem and lemma: \Cref{thm:fundamental,lem:helper}
\item Multiple mixed: \Cref{thm:fundamental,thm:completeness,lem:helper,lem:bounded}
\item With definition: \Cref{thm:fundamental,lem:helper,def:cauchy}
\end{itemize}

\section{Command Reference}

For practical use, here are the essential \texttt{cleveref} commands and their syntax:

\subsection{Basic Commands}

\begin{verbatim}
% Single reference
\Cref{label}              % "Theorem 1"
\cref{label}              % "theorem 1" (lowercase)

% Multiple references (same type)
\Cref{label1,label2}      % "Theorems 1 and 2"
\Cref{label1,label2,label3} % "Theorems 1, 2, and 3"

% Mixed type references
\Cref{thm:1,lem:1,def:1}  % "Theorem 1, Lemma 1, and Definition 1"
\end{verbatim}

\subsection{Range References}

\begin{verbatim}
% Consecutive references
\crefrange{label1}{label3}  % "theorems 1 to 3"
\Crefrange{label1}{label3}  % "Theorems 1 to 3"
\end{verbatim}

\subsection{Customization}

\begin{verbatim}
% At start references (beginning of sentence)
\Cref{label}              % Always capitalized

% Mid-sentence references  
\cref{label}              % Lowercase version
\end{verbatim}

\subsection{Special Cases}

\begin{verbatim}
% Subequations
\Cref{eq:system}          % References entire system
\Cref{eq:system-a}        % References individual part

% Page references (if needed)
\cpageref{label}          % "page 1"
\Cpageref{label}          % "Page 1"
\end{verbatim}

\section{Float References}

\subsection{Tables}

\begin{table}[htbp]
\centering
\caption{Comparison of convergence rates}
\label{tab:convergence}
\begin{tabular}{@{}lcc@{}}
\toprule
Method & Rate & Iterations \\
\midrule
Newton's & Quadratic & 5 \\
Bisection & Linear & 20 \\
Secant & Superlinear & 8 \\
\bottomrule
\end{tabular}
\end{table}

\begin{table}[htbp]
\centering
\caption{Function properties}
\label{tab:functions}
\begin{tabular}{@{}lccc@{}}
\toprule
Function & Continuous & Differentiable & Integrable \\
\midrule
$x^2$ & Yes & Yes & Yes \\
$|x|$ & Yes & No & Yes \\
$\lfloor x \rfloor$ & No & No & Yes \\
\bottomrule
\end{tabular}
\end{table}

Table references:
\begin{itemize}
\item Single table: \Cref{tab:convergence}
\item Multiple tables: \Cref{tab:convergence,tab:functions}
\end{itemize}

\subsection{Figures}

\begin{figure}[htbp]
\centering
\begin{tikzpicture}
\draw[->] (-2,0) -- (2,0) node[right] {$x$};
\draw[->] (0,-1) -- (0,2) node[above] {$y$};
\draw[blue, thick] (-1.5,1.125) parabola bend (0,0) (1.5,1.125);
\node[below] at (0,0) {$0$};
\end{tikzpicture}
\caption{Graph of $y = x^2$}
\label{fig:parabola}
\end{figure}

\begin{figure}[htbp]
\centering
\begin{tikzpicture}
\draw[->] (-2,0) -- (2,0) node[right] {$x$};
\draw[->] (0,-2) -- (0,2) node[above] {$y$};
\draw[red, thick] (-1.5,-1.5) -- (1.5,1.5);
\node[below] at (0,0) {$0$};
\end{tikzpicture}
\caption{Graph of $y = x$}
\label{fig:linear}
\end{figure}

Figure references:
\begin{itemize}
\item Single figure: \Cref{fig:parabola}
\item Multiple figures: \Cref{fig:parabola,fig:linear}
\end{itemize}

\subsection{Mixed Float References}

We can also mix different float types:
\begin{itemize}
\item Table and figure: \Cref{tab:convergence,fig:parabola}
\item Multiple mixed: \Cref{tab:convergence,tab:functions,fig:parabola,fig:linear}
\end{itemize}

\section{Range References}

Cleveref also supports range references using \texttt{\textbackslash crefrange}:

\begin{theorem}\label{thm:first-range}
First theorem in range.
\end{theorem}

\begin{theorem}\label{thm:second-range}
Second theorem in range.
\end{theorem}

\begin{theorem}\label{thm:third-range}
Third theorem in range.
\end{theorem}

We can reference ranges: \crefrange{thm:first-range}{thm:third-range} covers all three results.

\section{Conclusion}

This document demonstrates \texttt{cleveref}'s capabilities:
\begin{enumerate}
\item Automatic formatting for single and multiple references
\item Smart handling of mixed reference types  
\item Proper formatting of floats (tables and figures)
\item Range references for consecutive items
\item Command reference for practical use
\end{enumerate}

The combination of your paper style with \texttt{cleveref} provides a robust cross-referencing system that scales well from simple single references to mathematical environment combinations like \Cref{thm:fundamental,lem:helper}.

Key advantages of using \texttt{cleveref}:
\begin{itemize}
\item Reduces manual typing and formatting errors
\item Maintains consistency across the document
\item Automatically handles pluralization and punctuation
\item Works seamlessly with your paper style's theorem environments
\item Provides both capitalized (\texttt{\textbackslash Cref}) and lowercase (\texttt{\textbackslash cref}) variants
\end{itemize}

This document serves as both a demonstration and a quick reference for using \texttt{cleveref} effectively with your academic paper template.

\end{document}
